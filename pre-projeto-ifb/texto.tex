%% abtex2-modelo-relatorio-tecnico.tex, v<VERSION> laurocesar
%% Copyright 2012-2015 by abnTeX2 group at http://www.abntex.net.br/ 
%%
%% This work may be distributed and/or modified under the
%% conditions of the LaTeX Project Public License, either version 1.3
%% of this license or (at your option) any later version.
%% The latest version of this license is in
%%   http://www.latex-project.org/lppl.txt
%% and version 1.3 or later is part of all distributions of LaTeX
%% version 2005/12/01 or later.
%%
%% This work has the LPPL maintenance status `maintained'.
%% 
%% The Current Maintainer of this work is the abnTeX2 team, led
%% by Lauro César Araujo. Further information are available on 
%% http://www.abntex.net.br/
%%
%% This work consists of the files abntex2-modelo-relatorio-tecnico.tex,
%% abntex2-modelo-include-comandos and abntex2-modelo-references.bib
%%

% ------------------------------------------------------------------------
% ------------------------------------------------------------------------
% abnTeX2: Modelo de Relatório Técnico/Acadêmico em conformidade com 
% ABNT NBR 10719:2011 Informação e documentação - Relatório técnico e/ou
% científico - Apresentação
% Adaptado por Daniel Saad Nogueira Nunes para uso no IFB Taguatinga.
% ------------------------------------------------------------------------ 
% ------------------------------------------------------------------------


% Coloque opção para bacharelado ou Licenciatura
\documentclass[licenciatura]{pre-projeto-computacao}

% ---
% Informações de dados para CAPA e FOLHA DE ROSTO
% ---
\titulo{Título do Pré-projeto}
\autor{Virgulino Ferreira da Silva}
\orientador{Padre Cícero}
\coorientador[Coorientadora:]{Maria Bonita}

% Linha de pesquisa conforme tabela de áreas do CNPq
\areadepesquisa{1.03.01.01-1 Computabilidade e Modelos de Computação }
\local{Brasília, DF}
\data{2018}

\begin{document}

\selectlanguage{brazil}
\frenchspacing 
\imprimircapa
\imprimirfolhaderosto*




\section*{Introdução}
	Este modelo visa adaptar a classe ABNTeX2 para utilização no IFB Taguatinga como documento de Anteprojeto necessário para matrícula em TCC1.
	
	
	Neste modelo, apresentamos uma \textbf{sugestão} de seções adotadas, nela se enquadrado: Introdução, Revisão da Literatura, Justificativa, Proposta, Metodologia e Cronograma e Descrição Orçamentária do Projeto. O aluno e orientador podem adotar outra divisão se acharem conveniente.
	
	A bibliografia deve seguir o padrão estabelecido pela classe ABNTeX2 \cite{abntex2modelo-relatorio}.
	
	A introdução deverá contextualizar o problema e motivar o leitor.
\section*{Revisão da Literatura}

\section*{Justificativa}

\section*{Proposta}

\subsection*{Objetivos Gerais}
\subsection*{Objetivos Específicos}

\section*{Metodologia}

\section*{Cronograma e Descrição Orçamentária do Projeto}


\bibliography{bibliografia}
\end{document}
