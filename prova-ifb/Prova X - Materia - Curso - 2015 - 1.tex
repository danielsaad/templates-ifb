% Arquivo editável LaTeX para elaboração de provas. É baseado na classe
% exam do LaTeX.
% Copyright (C) 2018  Daniel Saad Nogueira Nunes (daniel.nunes@ifb.edu.br)

% This program is free software: you can redistribute it and/or modify
% it under the terms of the GNU General Public License as published by
% the Free Software Foundation, either version 3 of the License, or
% (at your option) any later version.

% This program is distributed in the hope that it will be useful,
% but WITHOUT ANY WARRANTY; without even the implied warranty of
% MERCHANTABILITY or FITNESS FOR A PARTICULAR PURPOSE.  See the
% GNU General Public License for more details.

% You should have received a copy of the GNU General Public License
% along with this program.  If not, see <http://www.gnu.org/licenses/>.


\documentclass[12pt,addpoints,a4paper]{exam}
\usepackage{amsmath,amssymb}
\usepackage{epigraph}
\usepackage[english,brazil]{babel}
\usepackage[utf8]{inputenc}
\usepackage[T1]{fontenc}
\usepackage{datetime}
\usepackage{enumerate}
\usepackage{url}
\usepackage{bussproofs}




%%para inclusão de figuras: jpg, pdf, mps, png
\usepackage[pdftex]{graphicx}
\usepackage{color}
\usepackage{cancel}%%\cancelto{<value>}{<expression>}
%\newcommand{\docdate}{\today}
%\renewcommand{\solutiontitle}{\noident\textbf{Solução:}}\par\noident
\pagestyle{headandfoot}
\runningheadrule
\footer{$\star$ Certifique-se de assinar todas as folhas de resposta.}{}%
{\iflastpage{}{}}
\extrafootheight{10mm}

\extrawidth{10mm}
\shadedsolutions
%\boxedpoints2
%\nopointsinmargin
\pointpoints{ ponto}{ pontos}
\pointname{ \points}
%\marginpointname{ \points}
\qformat{\textbf{\large{Questão} \thequestion} \quad (\thepoints)\hfill}

\hpword{Pontos:}
\vpword{Pontos}
\htword{Total}
\vtword{Total}
\vsword{Nota}
\vqword{Questão}


\newcommand{\insertquote}{\iflastpage{\vfill \epigraph{I have no idols. I admire work, dedication and competence}{Ayrton Senna}}{\hfill Continua \ldots}}
\newcommand{\data}{15 de junho de 2015}
\newcommand{\instituto}{Instituto Federal de Educação, Ciência e Tecnologia de Brasília}
\newcommand{\campus}{Câmpus Taguatinga}
\newcommand{\curso}{Ciência da Computação}
\newcommand{\disciplina}{Estrutura de Dados e Algoritmos}
\newcommand{\assunto}{Algoritmos, Alocação Dinâmica e Ordenação}
\newcommand{\professor}{Prof. Daniel Saad Nogueira Nunes}
\newcommand{\semestre}{1$^\circ$/2017}
\newcommand{\duracao}{Duração da prova: 100 minutos}
\newcommand{\prova}{Prova I}
\newcommand{\cabecalho}{
	
	\instituto{} -- \campus\\
	\curso{} -- \disciplina\\
	\prova{} -- \semestre{} -- \assunto\\
	\professor
}
\begin{document}

\begin{center}
	\includegraphics[scale=0.3]{IFBVertical.png}
	\cabecalho
\end{center}



\vspace{5mm}

\begin{minipage}[b]{1.0\linewidth}
	Aluno:\enspace\hbox to 110mm{\hrulefill}\\
	Matrícula:\enspace\hbox to 104mm{\hrulefill}\\
	Data: \data
\end{minipage}

\begin{center}
	\fbox{ \duracao}
\end{center}

\begin{center}
	Tabela de notas (uso exclusivo do professor)\\
	\addpoints
	\gradetable[v][questions]
\end{center}


\subsection*{Observações}
\begin{itemize}
	\item Esta prova tem o total de \numpages\ páginas (incluindo a capa) e \numquestions\ questões.
	\item O número total de pontos é \numpoints.
	\item Certifique-se de assinar todas as folhas de resposta bem como a capa da prova.
	\item Leia atentamente todas as questões da prova. A interpretação do problema é crucial para o desenvolvimento correto da resposta.
	\item Resoluções sem justificativa não serão consideradas.
	\item É vedado o uso de equipamentos eletrônicos, como celulares, notebooks entre outros.
	\item A prova será \textbf{anulada} e medidas disciplinares serão tomadas para os alunos que ``colarem'' durante a avaliação. 
	
\end{itemize}


\newpage

\begin{questions}
	
	\question[2] Qual a cor do céu?
	\question[2] Que dia é hoje?
	\question[2] Qual é o seu nome?
	\question[2] Quantos anos você tem?
	\question[2] De acordo com a aritmética:
	\begin{parts}
		\noaddpoints
		\part[1] 2+2?
		\part[1] 1+1?
		\addpoints
	\end{parts}
	{\insertquote}
	\extrafootheight{10mm}
\end{questions}
	
	

\end{document}
